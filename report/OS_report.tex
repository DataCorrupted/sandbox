%% OS II Project Report Latex file.
%% Completed By Yuyang Rong(rongyy@shanghaitech.edu.cn) and 
%% Jianxiong Cai(caijx@shanghaitech.edu.cn)
%%
%% To edit this file, please use indentions with tab size of 2.
%%

\documentclass[conference,compsoc]{IEEEtran}
\usepackage{cite}
\usepackage{listings}



% correct bad hyphenation here
\hyphenation{op-tical net-works semi-conduc-tor}

\begin{document}
\title{
	OS Course Project: Safebox\\ 
	Software fault isolation using syscall prevention approach
}


% author names and affiliations
% use a multiple column layout for up to three different
% affiliations
\author{
	\IEEEauthorblockN{Yuyang Rong}
	\IEEEauthorblockA{
		School of Information Science and Technology \\
		ShanghaiTech University \\
		Student ID: 69850764 \\
	}
\and
	\IEEEauthorblockN{Jianxiong Cai}
	\IEEEauthorblockA{
		School of Information Science and Technology \\
		ShanghaiTech University \\
		Student ID: 67771603 \\
	}
}

\maketitle

% As a general rule, do not put math, special symbols or citations
% in the abstract
\begin{abstract}
	How to test program, either for student homework or competition, is a tricky problem since the grading server have to protect itself from being hacked by malicious program submitted from Internet, and protect test cases from being poached by untrusted program. 
	To be general, a safe environment is needed for user to run untrusted program where user and his data can be safe. 
	Currently, there are two ways to create such an environment. One is to use container or virtual machine to create a vritual environment to run the untrusted program. The other is to track and limit the syscall made by untrusted program. There are pros and cons for both project.
	In this project, we implemented a sandbox for safely running untrusted program, using the second way. Besides, we develpe a rust lib for trace syscall of another process.
\end{abstract}

\section{Introduction}
	\subsection{Real life problem}
		\par
			How to test a program, either for student homework or competition, is a tricky problem since the grading server have to run the program in order to grade the program under most situations. On the other hand, because the student homework or competition code is submitted from Internet, and is likely to be buggy and may damage the operating system running it. Besides that, some program may try to poke the testcases during grading and try to send it to the author.
			Therefore, the student homework or competition code should be considered as untrusted program. The operating system need to protect itself from being damaged and prevent the program from poking testcases.
		\par
			As the result, a safe environment is needed for user to run untrusted program where user and his data can be safe. 
\section {Motivation}
	\subsection{Tread Model}
		\par
			As has mentioned before, the untrusted program may perform following operations, either intendically or accidentially to damage the system or poke testcases from grading.
			\begin{itemize}
				\item {Read files not owned to the current user}
				\item {Remove files owned by the current user, but should not be removed or changed by the program, like testcases}
				\item {%TODO, fill it in if there is any}
				\item {Read in testcase and send it out through Internet.}
			\end{itemize}
		\par
			As the result, there are a few requirements for the sandbox.
			\begin{itemize}
				\item {Protect files not owned by the user (uid)}
				\item {Prevent program from removing files owned by current user but should not be removed, like testcases}
				\item {Limited Internet access for the program}
			\end{itemize}
	\subsection{Current Slouations}
		% introduce the LXD and Janus paper here. TODO rewrite the following section
		\par
			Currently there are many mal-ware isolation applications, one of the most famous being \emph{lxd}. 
			We dived into \emph{lxd} and realized that it's designed based on the following principles:
		\par
			\emph{\textbf{
				An application can do little harm if its access to the underlying operating system is appropriately restricted.\cite{SecSandBox}
			}}
		\par
			Good as this principle is, it causes unintended inconvenience. 
			For example, Rust compiler have to check Github.com for extern crates if it is one of the dependences of the user program. 
			But to make sure the data is safe, in \emph{lxd}, network access is either entirely banned or allowed.
			Such situation can be really problematic.
		\par
			Thus we realized we need something better to help cope with the situation. 
			We want network, file system, etc be available without damaging user's computer or server. 
			That's how we started our project.
\section {Project Contribution}
\begin{itemize}
	\item{Implement Janus sandbox in Rust}
	\item{Implement the sandbox to allow untrusted program only can connect to primitted ip address}
	\item{Implement a Rust lib to trace another process, based on ptrace with Rust FFI (Foreign Function Interface)}
\end{itemize}
\section {Design}
	\subsection{Principles}
		\par
			So we came to our solution. As we have mentioned above, the LXD does not provide limited Internet access, instead, it either allow all Internet access or deny all. So we did not followed the priciple of container / virtual machine to implement the sandbox.
			\par
			Instead, we followed the priciple of Janus sandbox, because the assumption Janus sandbox made, as mentioned above, is quite concrete.
			\par
			In general, we will trace every syscall the untrusted program is issuing and decide if it's an allowance or a denial.
	\subsection {Alternative ways}
	%TODO put all alternative ways here and examplain why it is not so good as what we finally use.
	\subsection{Difficulties} %TODO says about the difficities for the current implement, TODO rewrite pending
		\subsubsection{Outnumbered}
			\par
				Nice as tracing every syscall sounds, it's hard as soon as we realized the real scale the problem is. 
			\par 
				The number of syscall are enormous, far beyond our expectation. 
				For example, in x86\_64 architecture, there are in total 328 syscalls. 
				\par
				Writing 328 different policies to deal with each syscall is is hard and laborious. Instead of doing that, because there are connections between some syscalls, like certain syscalls have to be performed before others.
				\par
				For example, each and every program have to \emph{ open() } a file before it can \emph{read()} or \emph{write()} it. 
				Or, \emph{ socket() } have to be established before a network connection can be possible.
			\par
			%% TODO kind of wired
				So we decided to chock the untrusted program on those syscalls, to put pressure on where it hurts. 
				For example, we allow \emph{ close() }, \emph{ read() } directly, but we carefully examine \emph{ open() }, whether the file is allowed, what it is opened for, etc.
			\par 
				In this way, we narrowed our work to much fewer syscalls like \emph{open()} and \emph{connect()}.
		\subsubsection{Killing a children}
			\par 
				Not much program nowadays do not do \emph{ fork() } anymore, even students' homework use this syscall here or there. 
				When a program \emph{ fork() }, tracing program do \emph{ fork() } too, and immediately begin to trace the child just forked by the untrusted program. 
				After these finishes, we will start both parent and child.
			\par
				The problem is, when a untrusted program \emph{kill()} it's child, or one of it's children \emph{kill()} another, the tracer can't tell the difference. 
				Tracer have no way to know the \emph{pid} about to be killed is another user program or one of one of tracee's children. 
				Tracer program have to do inter communication to know what really happened.
			\par
				Doing inter communication between tracers can be rather difficult can gains little, we then came up with a new solution that we change tracee's \emph{uid}, thus every tracee's child will have the same \emph{uid} and they can feel free to \emph{kill()} each other, but once they try to \emph{kill()} anyone else, operating system will stop them.


% the actual implementation part.
\section{Implementation}


	\subsection{ptrace}

		\par
			Linux provides a function:\\
				\begin{center}
				\emph{ptrace(request, pid, addr, data)}
				\end{center}
			This function can be called both by untrusted program or tracing program. 
			In user manual, the program being traced is called \emph{Tracee}, we will call it tracee from now on.
		\par 
			What ptrace does is that it stops syscall and have the privilege to examine memory, registers, etc.\ the tracee is using. 
			Every time the tracer calls this function, it will be put to wait. 
			It will wake up on two conditions: either tracee issues an syscall then OS wakes tracer up and continue doing the syscall until tracer says so by a \emph{ptrace(CONT, pid, void, void)} call or OS finished a syscall it wakes tracer up before it hands control to tracee.

	\subsection{Choice of architecture}

		\par
			ptrace may have the ability to alter registers, it don't necessarily have the ability to know the syscall the tracee is making. 
			Different architectures store syscall number and syscall return value in different registers. 
			For example in MIPS they are stored in \emph{\$a0} x86 \emph{\%eax}, x86\_64 \emph{\%rax}.
		\par
			MISP is not widely used in PC, it's mostly used in embedded processors and they don't need that much software isolation. 
			Besides, it rather an old architecture.
		\par 
			We considered making a sandbox on x86, but it's an old architecture too. 
			We totally give it up because we realized that x86\_64 can run x86 program too, so if we use x86\_64 architecture, only slight changes shall be made then sandbox can run on x86.
		\par 
			We decided we will use x86\_64 architecture since this is the most widely used architecture, easier for us to program when both of us are using x86\_64 architecture laptop.

	\subsection{Programming language}

		\par 
			\emph{ptrace()} is a C function, so we thought let's do it in C. 
			But after we carefully examined several programming languages we decided to use Rust.
		\subsubsection{C}
			\par
				C is the first programming language came into our mind. 
				But after we toyed a little with C and \emph{ptrace()} we realized that \emph{ptrace()} is rather a complicated function, we want to encapsulate it to make our life easier. 
				However, C is not OOP, so we abandoned it and started to consider C++.
		\subsubsection{C++}
			\par
				C++ is good at encapsulation, but too week at type system. 
				We build this sandbox to be safe, not to introduce more volubilities. 
				So instead of C++ we started looking Rust.
		\subsubsection{Rust}
			\par
				Rust have it's pros and cons too. 
				There is not \emph{ptrace()} implemented in Rust, we can only use this function through extern crate \emph{libc}, so we are forced to add a lot of \emph{unsafe} to our code. 
				However, life time system and ownership system allow us to code more confidently.

	\subsection{Encapsulation}
		\par 
			We found a finished ptrace library on crates.io, but it couldn't compile because of certain dependencies are missing. 
			We tried to fix it, but we failed because the lacking dependencies have been removed by Rust team.
		\par
			So we decide to write ourselves a library to do that job. 
			We created a structure contains a pid and other necessary components listed below:
			\begin{verbatim}
				pub struct Tracee {
					  pid: libc::pid_t,
					  allow_all: bool,
					  entry_flag : bool,			
					  last_syscall: u64,
					  ip_connected: Vec<String>,
					  file_opened: Vec<String>,
				}
			\end{verbatim}
		\par
			\emph{allow\_all} is a sign in case the user don't want any protection. 
			\emph{entry\_flag} shows if the tracee is exiting a syscall or entering. 
			The last two vectors are just logs used to tell tracee's behavior.
		\par
			We can initialize a Tracee by telling the command so that the constructor will call \emph{fork()} first and the child will call \emph{ptrace(TRACEME, pid, void, void)}, thus a program is set up and ready to go.(The program won't start until tracer says so.)
		\par
			We now provide interface for certain widely used functions, like \emph{take\_regs()} and \emph{do\_continue()}. 
			These functions do not require any arguments but pid, using \emph{ptrace()} to do that can be laborious, so we finished the interface.
	\subsection{The framework}
		\subsubsection{Reading configuration file}
			Currently Safebox have 2 configuration files, the ip address and file allowed to access.
			%% Ernest goes here.
		\subsubsection{Initialize the tracee}
			\par
				Good thing is this have been done in our own library, all we need to do is to take command from command line and put them into a vector, give that to constructor. 
			\par
				After construction, the tracee is halted at the first \emph{execvp()}, user have to manually start the tracee by calling \emph{tracee.do\_continue()}

		\subsubsection{Tracing according to syscall}
			\par
				The main program goes into a loop which will not break until the tracee is stopped because it exited instead of an issue of syscall.
				In the loop we will intercept every syscall, each twice.(In and out)
				Now we don't do any examine on return value of a syscall so we directly allow it by saying \emph{tracee.do\_continue)()}.
				However, if the tracee is entering a syscall, that's another story.
				We will make decisions depend on the syscall the tracee is making.
		\subsubsection{Allow or Deny}
			\par
				Currently we focus on two syscalls, \emph{open()} and \emph{connect()}
			\par
				In terms of \emph{open}, we only allow files inside current working directory, outside /home or specifically permitted in configuration file.
				We believe if tracee is toying files inside current working directory, little harm can be made.
				If tracee tries to tank files outside /home, operating system will prevent it.
				
		\subsection{Network}
			\par
				There are different ways to limit the access of certain program to network. 
				\emph{lxd}, the Linux Containers, limit the network access by configuring the whole container, all programs in one container will either have network access or not. 
				Taking advantage of syscall-level sandbox, every process is separate, thus monitoring one particular program's network connection is possible. 
				Besides, by implementing in this way, the sandbox can allow the untrusted program only be able to connect to certain IP address.
			\par
				There are four syscall a client program would use to make a network connection. 
			\begin{enumerate}
			\item socket
				\par
				get a new socket.
			\item connect
				\par
				connect the socket to a remote IP.
			\item receive from / send to
				\par
				communicating with the remote IP though the socket.
			\end{enumerate}
			\par 
				Because the connect syscall is the only syscall whose arguments contain the IP address, tracing this syscall can get the IP address, then the sandbox can decide whether this connection is allowed or not. 
			\par
				For example, if the untrusted program is only allowed to connect to Github.com, when the sandbox found the IP address which untrusted program is trying to connect to is beyond the white list, this syscall will be rejected.
			

% TODO rewrite the part (structure)
\section{Rust Experience}
	After finish this project, we now have a better feeling about Rust programming language. It's pros and cons we will list our feelings here.
	%% TODO reconstruct this
	\subsection{Pros}
		\par
			\subsubsection{Trait system} 
				It is really powerful, it allows us to add additional method to predefined types. 
				When doing file path parsing, a new method need to be add to String type, Trait system allows it.
			\subsubsecton{ Mutability system } 
				In C++, variables are assumed mutable. 
				Thus it is dangerous to return a pointer to a variable in a struct, because the user may accidentally change the variable and could easily corrupt the struct.
				Taking advantage of Rust, all variables are assumed immutable, thus the user can't change the variable. 
				In this way, struct can't be corrupted by buggy program.
	\subsection{Cons}
			\par
				Although Rust is said to be a system programming language, it still lack certain "system" interface.
				We have to heavily rely on \emph{libc} to do the programming and have tons of \emph{unsafe}s in it, which is not that desirable. 				

\section{Limitations \& Future Work}
	\subsection{File System}
		\par
			Thanks to Prof. Chen's advice, we tried to use user system in Linux to protect user's data. 
			If the file opened doesn't belong to tracee, the system will deny it and return -1 on \emph{open()} call.
			However, tracer needs to be root before it can change tracee's uid, although it worked but we are still concerned. 
			We are looking for ways to change uid without being root.
	\subsection{Network}
		\par
			Currently, the sandbox only support untrusted program making connect as a client. 
			If the untrusted program is a server application, the syscall it would use would use would be socket, bind, listen, and accept. 
		\par
			In the syscall "accept", the address of client on the other end of the connection would be provided, so it is reasonable to monitor this syscall to decide whether the connection should be allowed or not.


\section{Availability}

	Please check Github.com for our Safebox.
	%TODO add github link to here

\section*{Acknowledgment}
	We would like to thank Prof. Chen for his brilliant course and his guidance for our project. He provided us with such a great opportunity to do system programming and rust programming ourselves. 

%TODO cite papers and websites if necessary
\bibliographystyle{IEEEtran}
\bibliography{OS_cite}


% that's all folks
\end{document}


